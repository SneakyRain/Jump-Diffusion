\documentclass[paper.tex]{subfiles}

\begin{document}
    \section{Mathematical Preliminaries}
        \subsection{Poisson Process}
        A Poisson Process $ \boldsymbol{N} = \{N_{t}: t \geqslant 0 \} $ is a counting process, which is used to model occurence of random events such that their interarrival times are exponentially distributed with parametrer $ \lambda $.
        The number of events between any time interval $ (t, ~ t + \Delta t) $ follows poisson distribution, as given in Eq. (\ref{eq: poisson-distribution})
        \begin{equation}
            \mathbb{P}\{N=k\}= \dfrac{(\lambda \Delta t)^{k}}{k!} e^{-\lambda \Delta t}
            \label{eq: poisson-distribution}
        \end{equation}
        with,
        \begin{equation}
            \mathbb{E}\left[ N \right] = \lambda \Delta t
            \label{eq: poisson-mean}
        \end{equation}
          
        Poisson process is used to model a wide variety of process for e.g. such as customer arrival at a store, phone calls at a switchboard, or earthquakes etc.
        
        \subsection{Compound Poisson Process}
        Compound Poisson Process $\boldsymbol{J}=\left\{J_{t}: t \geqslant 0 \right\}$  are a generalilzation of poisson process.
        For these process the interarrival time between random events is still exponentially distributed, but the size of jump is not 1, rather it is randomly distributed according to some law $ F $.
        Commonly, the law $ F $ is assumed to be Normal Distribution.

        At any instant, the value of process is given by the sum of the jump size ($ Y_{k} $) till that instant.
        such that,
        \begin{equation}
            J_{t}=\sum_{k=1}^{N_{t}} ~ Y_{k}
            \label{eq: compound-poisson-process}
        \end{equation}
        where, $ N_{t} $ is number of poisson events till time $ t $.
        Therefore, a  compound Poisson process is a real-valued right-continuous process $\left(Z_{t}: t \geqslant 0\right)$ with the following properties.
        \begin{enumerate}
            \item Finitely many jumps: for all $\omega \in \Omega$, sampled path $t \mapsto J_{t}(\omega)$ has finitely many jumps in finite intervals.
            \item Independent increments: for all $t, ~ s \geqslant 0 ; ~ J_{t+s} - J_{t}$ is independent of past $\{J_{u}: u \leq t\}$,
            \item Stationary increments: for all $t, ~ s \geqslant 0$, distribution of $J_{t+s} - J_{t}$ depends only on $s$ and not on $t$.
        \end{enumerate}

        \subsection{Brownian Motion}
        A stochastic process $ \boldsymbol{W} = \{W_{t}: t \geqslant 0 t\} $ is called a standard Brownian Motion if the following properties hold:
        \begin{enumerate}
            \item $ W_{0} = 0 $.
            \item Independent increments: for every time point $ 0 \leqslant t_{1} \leqslant t_{2} \leqslant \hdots \leqslant t_{n} $, the increments of $ W $: $ W_{t_{n}} - W_{t_{n-1}} $, $ W_{t_{n-1}} - W_{t_{n-2}}$, $ \hdots $, $ W_{t_{1}} - W_{t_{0}} $ are independent random variable. 
            \item Normal distribution: For every $ \Delta t $, the increment $ \Delta W_{t} = W_{t + \Delta t} - W_{t} \sim \mathcal{N}(0, \Delta t)$
            % very time point $ 0 \leqslant t_{1} \leqslant t_{2} \leqslant \hdots \leqslant t_{n} $, the increments of $ W $: 
            % $ W_{t_{n}} - W_{t_{n-1}} $, $ W_{t_{n-1}} - W_{t_{n-2}}$, $ \hdots $, $ W_{t_{1}} - W_{t_{0}} $ are normally distributed with zero mean and variance $ ( t_{n} - t_{n-1} )$ , $ ( t_{n-1} - t_{n-2} )$, $ \hdots $, $ ( t_{1} - t_{0} ) $, respectively.
            \item Almost surely, the function $ t \to W_{t} $ is a continuous function for every t.
        \end{enumerate}
    
        \newpage
    \section{Code}
        \subsection{Monte Carlo}
        \lstinputlisting[language=Python]{../monte-carlo.py}
        \subsection{Parameter Estimation}
        \lstinputlisting[language=Python]{../parameters.py}
        \subsection{Comparison between Models}
        \lstinputlisting[language=Python]{../compare.py}

\end{document}