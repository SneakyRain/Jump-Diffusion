\documentclass[paper.tex]{subfiles}

\begin{document}
    Introduced in 1970s, Black-Scholes (BS) Model for option pricing was a groundbreaking work, due to which it won Nobel Prize in Economics.
    But, it was based on one key assumption i.e. stock prices follow Geometric Brownian Motion (GBM).
    Soon practitioners discovered that these assumptions although largely true, are not exactly correct.
    Since then the model has been generalized in many directions.
    One of the major shortcomings of BS model is that when some major incident occurs, then public reaction to the new information often causes jumps in stock prices.
    
    Consequently, this causes the real life distribution of stock returns to be sometimes significantly different from the normal distribution as theoretically predicted by BS model.
    This creates financial risk for investors worldwide, who are heavily involved in Stock Markets.
    To mitigate this risk Merton Jump Diffusion (MJD) model was introduced by researchers.
    MJD is a generalized version of BS model. It allows for poisson jumps with, with exponential waiting times and follows GBM meanwhile.
    In the further sections it is investigated that whether jump diffusion model is a better fit for stocks than Black-Scholes.

\end{document}